% Options for packages loaded elsewhere
\PassOptionsToPackage{unicode}{hyperref}
\PassOptionsToPackage{hyphens}{url}
%
\documentclass[
  brazilian,
]{book}
\usepackage{amsmath,amssymb}
\usepackage{lmodern}
\usepackage{ifxetex,ifluatex}
\ifnum 0\ifxetex 1\fi\ifluatex 1\fi=0 % if pdftex
  \usepackage[T1]{fontenc}
  \usepackage[utf8]{inputenc}
  \usepackage{textcomp} % provide euro and other symbols
\else % if luatex or xetex
  \usepackage{unicode-math}
  \defaultfontfeatures{Scale=MatchLowercase}
  \defaultfontfeatures[\rmfamily]{Ligatures=TeX,Scale=1}
\fi
% Use upquote if available, for straight quotes in verbatim environments
\IfFileExists{upquote.sty}{\usepackage{upquote}}{}
\IfFileExists{microtype.sty}{% use microtype if available
  \usepackage[]{microtype}
  \UseMicrotypeSet[protrusion]{basicmath} % disable protrusion for tt fonts
}{}
\makeatletter
\@ifundefined{KOMAClassName}{% if non-KOMA class
  \IfFileExists{parskip.sty}{%
    \usepackage{parskip}
  }{% else
    \setlength{\parindent}{0pt}
    \setlength{\parskip}{6pt plus 2pt minus 1pt}}
}{% if KOMA class
  \KOMAoptions{parskip=half}}
\makeatother
\usepackage{xcolor}
\IfFileExists{xurl.sty}{\usepackage{xurl}}{} % add URL line breaks if available
\IfFileExists{bookmark.sty}{\usepackage{bookmark}}{\usepackage{hyperref}}
\hypersetup{
  pdftitle={R na Cozinha},
  pdfauthor={Paulo Guilherme Pinheiro dos Santos},
  pdflang={pt-br},
  hidelinks,
  pdfcreator={LaTeX via pandoc}}
\urlstyle{same} % disable monospaced font for URLs
\usepackage{color}
\usepackage{fancyvrb}
\newcommand{\VerbBar}{|}
\newcommand{\VERB}{\Verb[commandchars=\\\{\}]}
\DefineVerbatimEnvironment{Highlighting}{Verbatim}{commandchars=\\\{\}}
% Add ',fontsize=\small' for more characters per line
\usepackage{framed}
\definecolor{shadecolor}{RGB}{248,248,248}
\newenvironment{Shaded}{\begin{snugshade}}{\end{snugshade}}
\newcommand{\AlertTok}[1]{\textcolor[rgb]{0.94,0.16,0.16}{#1}}
\newcommand{\AnnotationTok}[1]{\textcolor[rgb]{0.56,0.35,0.01}{\textbf{\textit{#1}}}}
\newcommand{\AttributeTok}[1]{\textcolor[rgb]{0.77,0.63,0.00}{#1}}
\newcommand{\BaseNTok}[1]{\textcolor[rgb]{0.00,0.00,0.81}{#1}}
\newcommand{\BuiltInTok}[1]{#1}
\newcommand{\CharTok}[1]{\textcolor[rgb]{0.31,0.60,0.02}{#1}}
\newcommand{\CommentTok}[1]{\textcolor[rgb]{0.56,0.35,0.01}{\textit{#1}}}
\newcommand{\CommentVarTok}[1]{\textcolor[rgb]{0.56,0.35,0.01}{\textbf{\textit{#1}}}}
\newcommand{\ConstantTok}[1]{\textcolor[rgb]{0.00,0.00,0.00}{#1}}
\newcommand{\ControlFlowTok}[1]{\textcolor[rgb]{0.13,0.29,0.53}{\textbf{#1}}}
\newcommand{\DataTypeTok}[1]{\textcolor[rgb]{0.13,0.29,0.53}{#1}}
\newcommand{\DecValTok}[1]{\textcolor[rgb]{0.00,0.00,0.81}{#1}}
\newcommand{\DocumentationTok}[1]{\textcolor[rgb]{0.56,0.35,0.01}{\textbf{\textit{#1}}}}
\newcommand{\ErrorTok}[1]{\textcolor[rgb]{0.64,0.00,0.00}{\textbf{#1}}}
\newcommand{\ExtensionTok}[1]{#1}
\newcommand{\FloatTok}[1]{\textcolor[rgb]{0.00,0.00,0.81}{#1}}
\newcommand{\FunctionTok}[1]{\textcolor[rgb]{0.00,0.00,0.00}{#1}}
\newcommand{\ImportTok}[1]{#1}
\newcommand{\InformationTok}[1]{\textcolor[rgb]{0.56,0.35,0.01}{\textbf{\textit{#1}}}}
\newcommand{\KeywordTok}[1]{\textcolor[rgb]{0.13,0.29,0.53}{\textbf{#1}}}
\newcommand{\NormalTok}[1]{#1}
\newcommand{\OperatorTok}[1]{\textcolor[rgb]{0.81,0.36,0.00}{\textbf{#1}}}
\newcommand{\OtherTok}[1]{\textcolor[rgb]{0.56,0.35,0.01}{#1}}
\newcommand{\PreprocessorTok}[1]{\textcolor[rgb]{0.56,0.35,0.01}{\textit{#1}}}
\newcommand{\RegionMarkerTok}[1]{#1}
\newcommand{\SpecialCharTok}[1]{\textcolor[rgb]{0.00,0.00,0.00}{#1}}
\newcommand{\SpecialStringTok}[1]{\textcolor[rgb]{0.31,0.60,0.02}{#1}}
\newcommand{\StringTok}[1]{\textcolor[rgb]{0.31,0.60,0.02}{#1}}
\newcommand{\VariableTok}[1]{\textcolor[rgb]{0.00,0.00,0.00}{#1}}
\newcommand{\VerbatimStringTok}[1]{\textcolor[rgb]{0.31,0.60,0.02}{#1}}
\newcommand{\WarningTok}[1]{\textcolor[rgb]{0.56,0.35,0.01}{\textbf{\textit{#1}}}}
\usepackage{longtable,booktabs,array}
\usepackage{calc} % for calculating minipage widths
% Correct order of tables after \paragraph or \subparagraph
\usepackage{etoolbox}
\makeatletter
\patchcmd\longtable{\par}{\if@noskipsec\mbox{}\fi\par}{}{}
\makeatother
% Allow footnotes in longtable head/foot
\IfFileExists{footnotehyper.sty}{\usepackage{footnotehyper}}{\usepackage{footnote}}
\makesavenoteenv{longtable}
\usepackage{graphicx}
\makeatletter
\def\maxwidth{\ifdim\Gin@nat@width>\linewidth\linewidth\else\Gin@nat@width\fi}
\def\maxheight{\ifdim\Gin@nat@height>\textheight\textheight\else\Gin@nat@height\fi}
\makeatother
% Scale images if necessary, so that they will not overflow the page
% margins by default, and it is still possible to overwrite the defaults
% using explicit options in \includegraphics[width, height, ...]{}
\setkeys{Gin}{width=\maxwidth,height=\maxheight,keepaspectratio}
% Set default figure placement to htbp
\makeatletter
\def\fps@figure{htbp}
\makeatother
\setlength{\emergencystretch}{3em} % prevent overfull lines
\providecommand{\tightlist}{%
  \setlength{\itemsep}{0pt}\setlength{\parskip}{0pt}}
\setcounter{secnumdepth}{5}
\usepackage{booktabs}


\ifxetex
  % Load polyglossia as late as possible: uses bidi with RTL langages (e.g. Hebrew, Arabic)
  \usepackage{polyglossia}
  \setmainlanguage[variant=brazilian]{portuguese}
\else
  \usepackage[main=brazilian]{babel}
% get rid of language-specific shorthands (see #6817):
\let\LanguageShortHands\languageshorthands
\def\languageshorthands#1{}
\fi
\ifluatex
  \usepackage{selnolig}  % disable illegal ligatures
\fi
\usepackage[]{natbib}
\bibliographystyle{apalike}

\title{R na Cozinha}
\author{Paulo Guilherme Pinheiro dos Santos}
\date{29 de junho de 2021}

\begin{document}
\maketitle

{
\setcounter{tocdepth}{1}
\tableofcontents
}
\hypertarget{apresentauxe7uxe3o}{%
\chapter{Apresentação}\label{apresentauxe7uxe3o}}

This is a \emph{sample} book written in \textbf{Markdown}. You can use anything that Pandoc's Markdown supports, e.g., a math equation \(a^2 + b^2 = c^2\).

The \textbf{bookdown} package can be installed from CRAN or Github:

Remember each Rmd file contains one and only one chapter, and a chapter is defined by the first-level heading \texttt{\#}.

To compile this example to PDF, you need XeLaTeX. You are recommended to install TinyTeX (which includes XeLaTeX): \url{https://yihui.org/tinytex/}.

\hypertarget{intro}{%
\chapter{Introdução}\label{intro}}

You can label chapter and section titles using \texttt{\{\#label\}} after them, e.g., we can reference Chapter \ref{intro}. If you do not manually label them, there will be automatic labels anyway, e.g., Chapter \ref{methods}.

Figures and tables with captions will be placed in \texttt{figure} and \texttt{table} environments, respectively.

\begin{Shaded}
\begin{Highlighting}[]
\FunctionTok{par}\NormalTok{(}\AttributeTok{mar =} \FunctionTok{c}\NormalTok{(}\DecValTok{4}\NormalTok{, }\DecValTok{4}\NormalTok{, .}\DecValTok{1}\NormalTok{, .}\DecValTok{1}\NormalTok{))}
\FunctionTok{plot}\NormalTok{(pressure, }\AttributeTok{type =} \StringTok{\textquotesingle{}b\textquotesingle{}}\NormalTok{, }\AttributeTok{pch =} \DecValTok{19}\NormalTok{)}
\end{Highlighting}
\end{Shaded}

\begin{figure}

{\centering \includegraphics[width=0.8\linewidth]{R_pg2_book_files/figure-latex/nice-fig-1} 

}

\caption{Here is a nice figure!}\label{fig:nice-fig}
\end{figure}

Reference a figure by its code chunk label with the \texttt{fig:} prefix, e.g., see Figure \ref{fig:nice-fig}. Similarly, you can reference tables generated from \texttt{knitr::kable()}, e.g., see Table \ref{tab:nice-tab}.

\begin{Shaded}
\begin{Highlighting}[]
\NormalTok{knitr}\SpecialCharTok{::}\FunctionTok{kable}\NormalTok{(}
  \FunctionTok{head}\NormalTok{(iris, }\DecValTok{20}\NormalTok{), }\AttributeTok{caption =} \StringTok{\textquotesingle{}Here is a nice table!\textquotesingle{}}\NormalTok{,}
  \AttributeTok{booktabs =} \ConstantTok{TRUE}
\NormalTok{)}
\end{Highlighting}
\end{Shaded}

\begin{table}

\caption{\label{tab:nice-tab}Here is a nice table!}
\centering
\begin{tabular}[t]{rrrrl}
\toprule
Sepal.Length & Sepal.Width & Petal.Length & Petal.Width & Species\\
\midrule
5.1 & 3.5 & 1.4 & 0.2 & setosa\\
4.9 & 3.0 & 1.4 & 0.2 & setosa\\
4.7 & 3.2 & 1.3 & 0.2 & setosa\\
4.6 & 3.1 & 1.5 & 0.2 & setosa\\
5.0 & 3.6 & 1.4 & 0.2 & setosa\\
\addlinespace
5.4 & 3.9 & 1.7 & 0.4 & setosa\\
4.6 & 3.4 & 1.4 & 0.3 & setosa\\
5.0 & 3.4 & 1.5 & 0.2 & setosa\\
4.4 & 2.9 & 1.4 & 0.2 & setosa\\
4.9 & 3.1 & 1.5 & 0.1 & setosa\\
\addlinespace
5.4 & 3.7 & 1.5 & 0.2 & setosa\\
4.8 & 3.4 & 1.6 & 0.2 & setosa\\
4.8 & 3.0 & 1.4 & 0.1 & setosa\\
4.3 & 3.0 & 1.1 & 0.1 & setosa\\
5.8 & 4.0 & 1.2 & 0.2 & setosa\\
\addlinespace
5.7 & 4.4 & 1.5 & 0.4 & setosa\\
5.4 & 3.9 & 1.3 & 0.4 & setosa\\
5.1 & 3.5 & 1.4 & 0.3 & setosa\\
5.7 & 3.8 & 1.7 & 0.3 & setosa\\
5.1 & 3.8 & 1.5 & 0.3 & setosa\\
\bottomrule
\end{tabular}
\end{table}

You can write citations, too. For example, we are using the \textbf{bookdown} package \citep{R-bookdown} in this sample book, which was built on top of R Markdown and \textbf{knitr} \citep{xie2015}.

\hypertarget{matemuxe1tica-buxe1sica-no-r}{%
\chapter{\texorpdfstring{Matemática básica no \texttt{R}}{Matemática básica no R}}\label{matemuxe1tica-buxe1sica-no-r}}

Aqui mostramos como executar algumas operações aritméticas básicas e algumas funções no \texttt{R}. Trazemos os códigos e ao final um vídeo explicativo com todas as operações listadas.

\hypertarget{aritmuxe9tica}{%
\section{Aritmética}\label{aritmuxe9tica}}

\begin{Shaded}
\begin{Highlighting}[]
\CommentTok{\# Soma:}
\DecValTok{1}\SpecialCharTok{+}\DecValTok{3}
\DecValTok{10}\SpecialCharTok{+}\DecValTok{2}

\CommentTok{\# Subtração:}
\DecValTok{5{-}2}
\DecValTok{10{-}2}
\DecValTok{2{-}10}

\CommentTok{\# Multiplicação;}
\DecValTok{2}\SpecialCharTok{*}\DecValTok{3}
\DecValTok{7}\SpecialCharTok{*}\DecValTok{4}

\CommentTok{\# Potenciação}
\DecValTok{2}\SpecialCharTok{\^{}}\DecValTok{3}
\DecValTok{4}\SpecialCharTok{\^{}}\DecValTok{4}
\DecValTok{2}\SpecialCharTok{**}\DecValTok{3}
\DecValTok{4}\SpecialCharTok{**}\DecValTok{4}

\CommentTok{\# Divisão;}
\DecValTok{8}\SpecialCharTok{/}\DecValTok{2}
\DecValTok{10}\SpecialCharTok{/}\DecValTok{3}

\CommentTok{\# Quociente da divisão; parte inteira: \%/\%}
\DecValTok{10}\SpecialCharTok{\%/\%}\DecValTok{3}

\CommentTok{\# Resto da divisão: \%\%}
\DecValTok{10}\SpecialCharTok{\%\%}\DecValTok{3}

\CommentTok{\# Módulo:}
\FunctionTok{abs}\NormalTok{(}\SpecialCharTok{{-}}\DecValTok{3}\NormalTok{)}
\FunctionTok{abs}\NormalTok{(}\DecValTok{8}\NormalTok{)}
\FunctionTok{abs}\NormalTok{(}\SpecialCharTok{{-}}\DecValTok{10}\NormalTok{)}

\CommentTok{\# Logarítmo:}
\FunctionTok{log}\NormalTok{(}\DecValTok{2}\NormalTok{)}
\FunctionTok{log}\NormalTok{(}\DecValTok{2}\NormalTok{,}\DecValTok{10}\NormalTok{)}
\FunctionTok{log10}\NormalTok{(}\DecValTok{2}\NormalTok{)}
\NormalTok{?log}
\FunctionTok{help}\NormalTok{(log)}
\FunctionTok{log}\NormalTok{(}\DecValTok{2}\NormalTok{,}\FunctionTok{exp}\NormalTok{(}\DecValTok{1}\NormalTok{))}
\NormalTok{log}

\CommentTok{\# Exponencial:}
\FunctionTok{exp}\NormalTok{(}\DecValTok{1}\NormalTok{)}
\FunctionTok{exp}\NormalTok{(}\DecValTok{3}\NormalTok{)}
\FunctionTok{exp}\NormalTok{(}\DecValTok{0}\NormalTok{)}

\CommentTok{\# Pi}
\NormalTok{pi}

\CommentTok{\# Funções Trigonométricas:}
\NormalTok{?sin}
\FunctionTok{sin}\NormalTok{(pi}\SpecialCharTok{/}\DecValTok{2}\NormalTok{)}
\FunctionTok{sinpi}\NormalTok{(}\DecValTok{1}\SpecialCharTok{/}\DecValTok{2}\NormalTok{)}
\FunctionTok{cos}\NormalTok{(pi}\SpecialCharTok{/}\DecValTok{2}\NormalTok{)}
\FunctionTok{cos}\NormalTok{(pi)}
\FunctionTok{cos}\NormalTok{(}\DecValTok{0}\NormalTok{)}
\FunctionTok{tan}\NormalTok{(pi}\SpecialCharTok{/}\DecValTok{4}\NormalTok{)}
\FunctionTok{sin}\NormalTok{(pi}\SpecialCharTok{/}\DecValTok{4}\NormalTok{)}\SpecialCharTok{/}\FunctionTok{cos}\NormalTok{(pi}\SpecialCharTok{/}\DecValTok{4}\NormalTok{)}

\CommentTok{\# Fatorial:}
\FunctionTok{factorial}\NormalTok{(}\DecValTok{4}\NormalTok{)}
\DecValTok{4}\SpecialCharTok{*}\DecValTok{3}\SpecialCharTok{*}\DecValTok{2}\SpecialCharTok{*}\DecValTok{1}

\CommentTok{\# Combinações:}
\FunctionTok{choose}\NormalTok{(}\DecValTok{10}\NormalTok{,}\DecValTok{2}\NormalTok{)}
\DecValTok{10}\SpecialCharTok{*}\DecValTok{9}\SpecialCharTok{/}\FunctionTok{factorial}\NormalTok{(}\DecValTok{2}\NormalTok{)}
\FunctionTok{factorial}\NormalTok{(}\DecValTok{10}\NormalTok{)}\SpecialCharTok{/}\NormalTok{(}\FunctionTok{factorial}\NormalTok{(}\DecValTok{10{-}2}\NormalTok{)}\SpecialCharTok{*}\FunctionTok{factorial}\NormalTok{(}\DecValTok{2}\NormalTok{))}

\CommentTok{\# Somatórios:}
\NormalTok{x }\OtherTok{=} \DecValTok{3}\SpecialCharTok{:}\DecValTok{13}
\NormalTok{x}
\FunctionTok{sum}\NormalTok{(x)}
\FunctionTok{cumsum}\NormalTok{(x)}
\FunctionTok{max}\NormalTok{(}\FunctionTok{cumsum}\NormalTok{(x))}
\NormalTok{x }\SpecialCharTok{|}\ErrorTok{\textgreater{}} \FunctionTok{cumsum}\NormalTok{() }\SpecialCharTok{|}\ErrorTok{\textgreater{}} \FunctionTok{max}\NormalTok{()}

\CommentTok{\# Produtórios:}
\NormalTok{x }\SpecialCharTok{|}\ErrorTok{\textgreater{}} \FunctionTok{prod}\NormalTok{()}
\FunctionTok{prod}\NormalTok{(x)}
\end{Highlighting}
\end{Shaded}

\begin{itemize}
\tightlist
\item
  \emph{Link} da aula: \href{https://youtu.be/QDLqS3y5u7Q}{Matemática Básica no R}.
\end{itemize}

\hypertarget{operador-pipe}{%
\section{\texorpdfstring{Operador \emph{Pipe}}{Operador Pipe}}\label{operador-pipe}}

Aqui temos um exemplo básico da utilização do operador \emph{pipe} \texttt{\textbar{}\textgreater{}}disponível no \texttt{R} a partir deste ano de 2021.`

\begin{itemize}
\tightlist
\item
  \emph{Link} da aula: \href{https://youtu.be/n27-lBxdfFg}{Operador Pipe}.
\end{itemize}

\hypertarget{visualizauxe7uxe3o-de-dados}{%
\chapter{Visualização de dados}\label{visualizauxe7uxe3o-de-dados}}

Neste capítulo apresentaremos como utilizar o pacote \texttt{ggplot2} para visualização de dados.

\hypertarget{a-gramuxe1tica-dos-gruxe1ficos}{%
\chapter{A gramática dos gráficos}\label{a-gramuxe1tica-dos-gruxe1ficos}}

Some \emph{significant} applications are demonstrated in this chapter.

\hypertarget{example-one}{%
\section{Example one}\label{example-one}}

\hypertarget{example-two}{%
\section{Example two}\label{example-two}}

\hypertarget{considerauxe7uxf5es-finais}{%
\chapter{Considerações Finais}\label{considerauxe7uxf5es-finais}}

We have finished a nice book.

\renewcommand{\bibname}{Referências}
\addcontentsline{toc}{chapter}{Referências}

  \bibliography{book.bib,packages.bib}

\end{document}
